% Options for packages loaded elsewhere
% Options for packages loaded elsewhere
\PassOptionsToPackage{unicode}{hyperref}
\PassOptionsToPackage{hyphens}{url}
\PassOptionsToPackage{dvipsnames,svgnames,x11names}{xcolor}
%
\documentclass[
  single column]{article}
\usepackage{xcolor}
\usepackage[top=30mm,left=25mm,heightrounded,headsep=22pt,headheight=11pt,footskip=33pt,ignorehead,ignorefoot]{geometry}
\usepackage{amsmath,amssymb}
\setcounter{secnumdepth}{-\maxdimen} % remove section numbering
\usepackage{iftex}
\ifPDFTeX
  \usepackage[T1]{fontenc}
  \usepackage[utf8]{inputenc}
  \usepackage{textcomp} % provide euro and other symbols
\else % if luatex or xetex
  \usepackage{unicode-math} % this also loads fontspec
  \defaultfontfeatures{Scale=MatchLowercase}
  \defaultfontfeatures[\rmfamily]{Ligatures=TeX,Scale=1}
\fi
\usepackage[]{libertinus}
\ifPDFTeX\else
  % xetex/luatex font selection
\fi
% Use upquote if available, for straight quotes in verbatim environments
\IfFileExists{upquote.sty}{\usepackage{upquote}}{}
\IfFileExists{microtype.sty}{% use microtype if available
  \usepackage[]{microtype}
  \UseMicrotypeSet[protrusion]{basicmath} % disable protrusion for tt fonts
}{}
\makeatletter
\@ifundefined{KOMAClassName}{% if non-KOMA class
  \IfFileExists{parskip.sty}{%
    \usepackage{parskip}
  }{% else
    \setlength{\parindent}{0pt}
    \setlength{\parskip}{6pt plus 2pt minus 1pt}}
}{% if KOMA class
  \KOMAoptions{parskip=half}}
\makeatother
% Make \paragraph and \subparagraph free-standing
\makeatletter
\ifx\paragraph\undefined\else
  \let\oldparagraph\paragraph
  \renewcommand{\paragraph}{
    \@ifstar
      \xxxParagraphStar
      \xxxParagraphNoStar
  }
  \newcommand{\xxxParagraphStar}[1]{\oldparagraph*{#1}\mbox{}}
  \newcommand{\xxxParagraphNoStar}[1]{\oldparagraph{#1}\mbox{}}
\fi
\ifx\subparagraph\undefined\else
  \let\oldsubparagraph\subparagraph
  \renewcommand{\subparagraph}{
    \@ifstar
      \xxxSubParagraphStar
      \xxxSubParagraphNoStar
  }
  \newcommand{\xxxSubParagraphStar}[1]{\oldsubparagraph*{#1}\mbox{}}
  \newcommand{\xxxSubParagraphNoStar}[1]{\oldsubparagraph{#1}\mbox{}}
\fi
\makeatother


\usepackage{longtable,booktabs,array}
\usepackage{calc} % for calculating minipage widths
% Correct order of tables after \paragraph or \subparagraph
\usepackage{etoolbox}
\makeatletter
\patchcmd\longtable{\par}{\if@noskipsec\mbox{}\fi\par}{}{}
\makeatother
% Allow footnotes in longtable head/foot
\IfFileExists{footnotehyper.sty}{\usepackage{footnotehyper}}{\usepackage{footnote}}
\makesavenoteenv{longtable}
\usepackage{graphicx}
\makeatletter
\newsavebox\pandoc@box
\newcommand*\pandocbounded[1]{% scales image to fit in text height/width
  \sbox\pandoc@box{#1}%
  \Gscale@div\@tempa{\textheight}{\dimexpr\ht\pandoc@box+\dp\pandoc@box\relax}%
  \Gscale@div\@tempb{\linewidth}{\wd\pandoc@box}%
  \ifdim\@tempb\p@<\@tempa\p@\let\@tempa\@tempb\fi% select the smaller of both
  \ifdim\@tempa\p@<\p@\scalebox{\@tempa}{\usebox\pandoc@box}%
  \else\usebox{\pandoc@box}%
  \fi%
}
% Set default figure placement to htbp
\def\fps@figure{htbp}
\makeatother


% definitions for citeproc citations
\NewDocumentCommand\citeproctext{}{}
\NewDocumentCommand\citeproc{mm}{%
  \begingroup\def\citeproctext{#2}\cite{#1}\endgroup}
\makeatletter
 % allow citations to break across lines
 \let\@cite@ofmt\@firstofone
 % avoid brackets around text for \cite:
 \def\@biblabel#1{}
 \def\@cite#1#2{{#1\if@tempswa , #2\fi}}
\makeatother
\newlength{\cslhangindent}
\setlength{\cslhangindent}{1.5em}
\newlength{\csllabelwidth}
\setlength{\csllabelwidth}{3em}
\newenvironment{CSLReferences}[2] % #1 hanging-indent, #2 entry-spacing
 {\begin{list}{}{%
  \setlength{\itemindent}{0pt}
  \setlength{\leftmargin}{0pt}
  \setlength{\parsep}{0pt}
  % turn on hanging indent if param 1 is 1
  \ifodd #1
   \setlength{\leftmargin}{\cslhangindent}
   \setlength{\itemindent}{-1\cslhangindent}
  \fi
  % set entry spacing
  \setlength{\itemsep}{#2\baselineskip}}}
 {\end{list}}
\usepackage{calc}
\newcommand{\CSLBlock}[1]{\hfill\break\parbox[t]{\linewidth}{\strut\ignorespaces#1\strut}}
\newcommand{\CSLLeftMargin}[1]{\parbox[t]{\csllabelwidth}{\strut#1\strut}}
\newcommand{\CSLRightInline}[1]{\parbox[t]{\linewidth - \csllabelwidth}{\strut#1\strut}}
\newcommand{\CSLIndent}[1]{\hspace{\cslhangindent}#1}



\setlength{\emergencystretch}{3em} % prevent overfull lines

\providecommand{\tightlist}{%
  \setlength{\itemsep}{0pt}\setlength{\parskip}{0pt}}



 


\usepackage{booktabs}
\usepackage{longtable}
\usepackage{array}
\usepackage{multirow}
\usepackage{wrapfig}
\usepackage{float}
\usepackage{colortbl}
\usepackage{pdflscape}
\usepackage{tabu}
\usepackage{threeparttable}
\usepackage{threeparttablex}
\usepackage[normalem]{ulem}
\usepackage{makecell}
\usepackage{xcolor}
\let\oldtabular\tabular
\renewcommand{\tabular}{\small\oldtabular}
\setlength{\tabcolsep}{4pt}
\makeatletter
\@ifpackageloaded{caption}{}{\usepackage{caption}}
\AtBeginDocument{%
\ifdefined\contentsname
  \renewcommand*\contentsname{Table of contents}
\else
  \newcommand\contentsname{Table of contents}
\fi
\ifdefined\listfigurename
  \renewcommand*\listfigurename{List of Figures}
\else
  \newcommand\listfigurename{List of Figures}
\fi
\ifdefined\listtablename
  \renewcommand*\listtablename{List of Tables}
\else
  \newcommand\listtablename{List of Tables}
\fi
\ifdefined\figurename
  \renewcommand*\figurename{Figure}
\else
  \newcommand\figurename{Figure}
\fi
\ifdefined\tablename
  \renewcommand*\tablename{Table}
\else
  \newcommand\tablename{Table}
\fi
}
\@ifpackageloaded{float}{}{\usepackage{float}}
\floatstyle{ruled}
\@ifundefined{c@chapter}{\newfloat{codelisting}{h}{lop}}{\newfloat{codelisting}{h}{lop}[chapter]}
\floatname{codelisting}{Listing}
\newcommand*\listoflistings{\listof{codelisting}{List of Listings}}
\makeatother
\makeatletter
\makeatother
\makeatletter
\@ifpackageloaded{caption}{}{\usepackage{caption}}
\@ifpackageloaded{subcaption}{}{\usepackage{subcaption}}
\makeatother
\usepackage{bookmark}
\IfFileExists{xurl.sty}{\usepackage{xurl}}{} % add URL line breaks if available
\urlstyle{same}
\hypersetup{
  pdftitle={Your Title},
  pdfauthor={YOUR NAME},
  pdfkeywords={Causal Inference, Cross-validation, \ldots{}},
  colorlinks=true,
  linkcolor={blue},
  filecolor={Maroon},
  citecolor={Blue},
  urlcolor={Blue},
  pdfcreator={LaTeX via pandoc}}


\title{Your Title}
\author{YOUR NAME}
\date{2025-05-27}
\begin{document}
\maketitle
\begin{abstract}
\textbf{Background}: (Brief few sentences) \textbf{Objectives}: 1.
Estimate the causal effect of YOUR EXPOSURE on YOUR OUTCOMES measured
one year later. 2. Evaluate whether these effects vary across the
population. 3. Provide policy guidance on which individuals might
benefit most. \textbf{Method}: We conducted a three-wave retrospective
cohort study (waves XX-XXX, October XXXX--October XXXX) using data
\emph{SIMULATED} from the New Zealand Attitudes and Values Study, a
nationally representative panel. Participants were eligible if they
participated in the NZAVS in the baseline wave (XXXX,\ldots). We defined
the exposure as (XXXX \textgreater{} NUMBER on a 1-7 Likert Scale (1 =
yes, 0 = no)). To address attrition, we applied inverse probability of
censoring weights; to improve external validity, we applied weights to
the population distribution of Age, Ethnicity, and Gender. We computed
expected mean outcomes for the population in each exposure condition
(high XXXX/low XXXXX). Under standard causal assumptions of
unconfoundedness, the contrast provides an unbiased average treatment
effect. We then used causal forests to detect heterogeneity in these
effects and employed policy tree algorithms to identify individuals
(``strong responders'') likely to experience the greatest benefits.
\textbf{Results}: Increasing XXXXX leads to XXXXX. Heterogeneous
responses to (e.g.~\emph{Forgiveness}, \emph{Personal Well-Being}, and
\emph{Life-Satisfaction}\ldots) reveal structural variability in
subpopulations\ldots{} \textbf{Implications}: (Brief few sentences)
\textbf{Keywords}: \emph{Causal Inference}; \emph{Cross-validation};
\emph{Distress}; \emph{Employment}; \emph{Longitudinal}; \emph{Machine
sLearning}; \emph{Religion}; \emph{Semi-parametric}; \emph{Targeted
Learning}.
\end{abstract}


\newpage{}

\subsection{Introduction}\label{introduction}

\textbf{Your place to shine here}

\subsection{Method}\label{method}

\newpage{}

\subsection{Results}\label{results}

\subsubsection{Average Treatement
Effects}\label{average-treatement-effects}

\begin{figure}

\centering{

\pandocbounded{\includegraphics[keepaspectratio]{initial_quarto_document_files/figure-pdf/fig-ate-1.pdf}}

}

\caption{\label{fig-ate}Average Treatment Effects on Multi-dimensional
Wellbeing}

\end{figure}%

\newpage{}

\begin{longtable}[]{@{}
  >{\raggedright\arraybackslash}p{(\linewidth - 10\tabcolsep) * \real{0.3415}}
  >{\raggedright\arraybackslash}p{(\linewidth - 10\tabcolsep) * \real{0.1220}}
  >{\raggedright\arraybackslash}p{(\linewidth - 10\tabcolsep) * \real{0.1220}}
  >{\raggedright\arraybackslash}p{(\linewidth - 10\tabcolsep) * \real{0.1220}}
  >{\raggedright\arraybackslash}p{(\linewidth - 10\tabcolsep) * \real{0.1220}}
  >{\raggedright\arraybackslash}p{(\linewidth - 10\tabcolsep) * \real{0.1707}}@{}}

\caption{\label{tbl-outcomes}Average Treatment Effects on
Multi-dimensional Wellbeing}

\tabularnewline

\toprule\noalign{}
\begin{minipage}[b]{\linewidth}\raggedright
Outcome
\end{minipage} & \begin{minipage}[b]{\linewidth}\raggedright
ATE
\end{minipage} & \begin{minipage}[b]{\linewidth}\raggedright
2.5 \%
\end{minipage} & \begin{minipage}[b]{\linewidth}\raggedright
97.5 \%
\end{minipage} & \begin{minipage}[b]{\linewidth}\raggedright
E-Value
\end{minipage} & \begin{minipage}[b]{\linewidth}\raggedright
E-Value bound
\end{minipage} \\
\midrule\noalign{}
\endhead
\bottomrule\noalign{}
\endlastfoot
\textbf{Social Belonging} & \textbf{0.133} & \textbf{0.09} &
\textbf{0.177} & \textbf{1.51} & \textbf{1.39} \\
\textbf{Neighbourhood Community} & \textbf{0.114} & \textbf{0.07} &
\textbf{0.159} & \textbf{1.458} & \textbf{1.328} \\
\textbf{Self Esteem} & \textbf{0.099} & \textbf{0.06} & \textbf{0.139} &
\textbf{1.415} & \textbf{1.299} \\
\textbf{Social Support} & \textbf{0.098} & \textbf{0.055} &
\textbf{0.141} & \textbf{1.413} & \textbf{1.284} \\
\textbf{Meaning: Purpose} & \textbf{0.104} & \textbf{0.053} &
\textbf{0.154} & \textbf{1.43} & \textbf{1.278} \\
\textbf{Meaning: Sense} & \textbf{0.094} & \textbf{0.043} &
\textbf{0.145} & \textbf{1.401} & \textbf{1.244} \\
Depression & -0.065 & -0.112 & -0.017 & 1.315 & 1.146 \\
Anxiety & -0.06 & -0.104 & -0.016 & 1.3 & 1.141 \\
Life Satisfaction & 0.056 & 0.012 & 0.1 & 1.287 & 1.121 \\
Personal Well-being Index & 0.055 & 0.013 & 0.098 & 1.284 & 1.116 \\
Rumination & -0.051 & -0.102 & -0.001 & 1.271 & 1.012 \\
Hours of Exercise (log) & -0.02 & -0.074 & 0.034 & 1.155 & 1 \\

\end{longtable}

\newpage{}

\newpage{}

\subsubsection{Heterogeneous Treatment
Effects}\label{results-qini-curve}

Table~\ref{tbl-qini} presents results for our Qini curve analysis at
different spend rates.

\textbf{?@fig-qini-1} presents results for reliable Qini results

\newpage{}

\subsubsection{Decision Rules (Who is Most Sensitive to
Treatment?)}\label{decision-rules-who-is-most-sensitive-to-treatment}

\newpage{}

\newpage{}

\newpage{}

\newpage{}

\newpage{}

\newpage{}

\newpage{}

\newpage{}

\newpage{}

\subsection{Discussion}\label{discussion}

\newpage{}

\subsection{Appendix A: Measures}\label{appendix-measures}

\subsubsection{Measures}\label{measures}

\paragraph{Baseline Covariate
Measures}\label{baseline-covariate-measures}

\subsubsection{Baseline Covariates}\label{baseline-covariates}

\paragraph{Age}\label{age}

\emph{What is your date of birth?}

We asked participants' ages in an open-ended question (``What is your
age?'' or ``What is your date of birth'').
(\citeproc{ref-sibley2021}{Sibley, 2021})

\paragraph{Agreeableness}\label{agreeableness}

\emph{I sympathize with others' feelings.} \emph{I am not interested in
other people's problems.} \emph{I feel others' emotions.} \emph{I am not
really interested in others (reversed).}

Mini-IPIP6 Agreeableness dimension: (i) I sympathize with others'
feelings. (ii) I am not interested in other people's problems. (r) (iii)
I feel others' emotions. (iv) I am not really interested in others. (r)
(\citeproc{ref-sibley2011}{Sibley et al., 2011})

\paragraph{Alcohol Frequency}\label{alcohol-frequency}

\emph{``How often do you have a drink containing alcohol?''}

Participants could chose between the following responses: `(1 = Never -
I don't drink, 2 = Monthly or less, 3 = Up to 4 times a month, 4 = Up to
3 times a week, 5 = 4 or more times a week, 6 = Don't know)'
(\citeproc{ref-Ministry_of_Health_2013}{Health, 2013})

\paragraph{Alcohol Intensity}\label{alcohol-intensity}

\emph{``How many drinks containing alcohol do you have on a typical day
when drinking alcohol? (number of drinks on a typical day when
drinking)''}

Participants responded using an open-ended box.
(\citeproc{ref-Ministry_of_Health_2013}{Health, 2013})

\paragraph{Social Belonging}\label{social-belonging}

\emph{Know that people in my life accept and value me.} \emph{Feel like
an outsider (reversed).} \emph{Know that people around me share my
attitudes and beliefs.}

We assessed felt belongingness with three items adapted from the Sense
of Belonging Instrument (Hagerty \& Patusky, 1995): (1) ``Know that
people in my life accept and value me''; (2) ``Feel like an outsider'';
(3) ``Know that people around me share my attitudes and beliefs''.
Participants responded on a scale from 1 (Very Inaccurate) to 7 (Very
Accurate). The second item was reversely coded.
(\citeproc{ref-hagerty1995}{Hagerty \& Patusky, 1995})

\paragraph{Born in Nz}\label{born-in-nz}

\emph{Where were you born? (please be specific, e.g., which town/city?)}

Coded binary (1 = New Zealand; 0 = elsewhere.)
(\citeproc{ref-sibley2021}{Sibley, 2021})

\paragraph{Conscientiousness}\label{conscientiousness}

\emph{I get chores done right away.} \emph{I like order.} \emph{I make a
mess of things.} \emph{I often forget to put things back in their proper
place.}

Mini-IPIP6 Conscientiousness dimension: (i) I get chores done right
away. (ii) I like order. (iii) I make a mess of things. (r) (iv) I often
forget to put things back in their proper place. (r)
(\citeproc{ref-sibley2011}{Sibley et al., 2011})

\paragraph{Education Level}\label{education-level}

\emph{What is your highest level of qualification?}

We asked participants, ``What is your highest level of qualification?''.
We coded participans highest finished degree according to the New
Zealand Qualifications Authority. Ordinal-Rank 0-10 NZREG codes (with
overseas school qualifications coded as Level 3, and all other ancillary
categories coded as missing) (\citeproc{ref-sibley2021}{Sibley, 2021})

\paragraph{Employed}\label{employed}

\emph{Are you currently employed (This includes self-employed of casual
work)?}

Binary response: (0 = No, 1 = Yes)
(\citeproc{ref-statsnz_ssga18}{Statistics New Zealand, 2017})

\paragraph{Ethnicity}\label{ethnicity}

\emph{Which ethnic group(s) do you belong to?}

Coded string: (1 = New Zealand European; 2 = Māori; 3 = Pacific; 4 =
Asian) (\citeproc{ref-statsnz_ssga18}{Statistics New Zealand, 2017})

\paragraph{Disability Status}\label{disability-status}

\emph{Do you have a health condition or disability that limits you and
that has lasted for 6+ months?}

We assessed disability with a one-item indicator adapted from Verbrugge
(1997). It asks, ``Do you have a health condition or disability that
limits you and that has lasted for 6+ months?'' (1 = Yes, 0 = No).
(\citeproc{ref-verbrugge1997}{Verbrugge, 1997})

\paragraph{Log Hours with Children}\label{log-hours-with-children}

\emph{Hours spent\ldots looking after children.}

We took the natural log of the response + 1.
(\citeproc{ref-sibley2011}{Sibley et al., 2011})

\paragraph{Log Hours Commuting}\label{log-hours-commuting}

\emph{Hours spent\ldots travelling/commuting.}

We took the natural log of the response + 1.
(\citeproc{ref-sibley2021}{Sibley, 2021})

\paragraph{Log Hours of Exercise}\label{log-hours-of-exercise}

\emph{Hours spent\ldots exercising/physical activity.}

We took the natural log of the response + 1.
(\citeproc{ref-sibley2011}{Sibley et al., 2011})

\paragraph{Log Hours on Housework}\label{log-hours-on-housework}

\emph{Hours spent\ldots housework/cooking.}

We took the natural log of the response + 1.
(\citeproc{ref-sibley2011}{Sibley et al., 2011})

\paragraph{Log Household Income}\label{log-household-income}

\emph{Please estimate your total household income (before tax) for the
year XXXX.}

We took the natural log of the response + 1.
(\citeproc{ref-sibley2021}{Sibley, 2021})

\paragraph{Male}\label{male}

\emph{We asked participants' gender in an open-ended question: ``what is
your gender?''}

Here, we coded all those who responded as Male as 1, and those who did
not as 0. (\citeproc{ref-fraser_coding_2020}{Fraser et al., 2020})

\paragraph{Neuroticism}\label{neuroticism}

\emph{I have frequent mood swings.} \emph{I am relaxed most of the time
(reversed).} \emph{I get upset easily.} \emph{I seldom feel blue
(reversed).}

Mini-IPIP6 Neuroticism dimension: (i) I have frequent mood swings. (ii)
I am relaxed most of the time. (r) (iii) I get upset easily. (iv) I
seldom feel blue. (r) (\citeproc{ref-sibley2011}{Sibley et al., 2011})

\paragraph{Non Heterosexual}\label{non-heterosexual}

\emph{How would you describe your sexual orientation? (e.g.,
heterosexual, homosexual, straight, gay, lesbian, bisexual, etc.)}

Open-ended question, coded as binary (not heterosexual = 1).
(\citeproc{ref-greaves2017diversity}{Greaves et al., 2017})

\paragraph{Nz Deprivation Index}\label{nz-deprivation-index}

\emph{New Zealand Deprivation - Decile Index - Using 2018 Census Data}

Numerical: (1-10) (\citeproc{ref-atkinson2019}{Atkinson et al., 2019})

\paragraph{Occupational Prestige
Index}\label{occupational-prestige-index}

\emph{We assessed occupational prestige and status using the New Zealand
Socio-economic Index 13 (NZSEI-13).}

This index uses the income, age, and education of a reference group, in
this case, the 2013 New Zealand census, to calculate a score for each
occupational group. Scores range from 10 (Lowest) to 90 (Highest). This
list of index scores for occupational groups was used to assign each
participant a NZSEI-13 score based on their occupation.
(\citeproc{ref-fahy2017}{Fahy et al., 2017})

\paragraph{Openness}\label{openness}

\emph{I have a vivid imagination.} \emph{I have difficulty understanding
abstract ideas (reversed).} \emph{I do not have a good imagination
(reversed).} \emph{I am not interested in abstract ideas (reversed).}

Mini-IPIP6 Openness to Experience dimension: (i) I have a vivid
imagination. (ii) I have difficulty understanding abstract ideas. (r)
(iii) I do not have a good imagination. (r) (iv) I am not interested in
abstract ideas. (r) (\citeproc{ref-sibley2011}{Sibley et al., 2011})

\paragraph{Parent}\label{parent}

\emph{If you are a parent, in which year was your eldest child born?}

Parents were coded as 1, while the others were coded as 0.
(\citeproc{ref-sibley2021}{Sibley, 2021})

\paragraph{Has Partner}\label{has-partner}

\emph{What is your relationship status? (e.g., single, married,
de-facto, civil union, widowed, living together, etc.)}

Coded as binary (has partner = 1). (\citeproc{ref-sibley2021}{Sibley,
2021})

\paragraph{Political Conservatism}\label{political-conservatism}

\emph{Please rate how politically liberal versus conservative you see
yourself as being.}

Ordinal response: (1 = Extremely Liberal, 7 = Extremely Conservative)
(\citeproc{ref-jost_end_2006-1}{Jost, 2006})

\paragraph{Major Religions}\label{major-religions}

\emph{Do you identify with a religion and/or spiritual group?
--\textgreater{} (If yes\ldots)--\textgreater{} What religion or
spiritual group?}

Open-ended (string). Coded from New Zealand Census Categories. Levels
are: ``Not Religious'',``Anglican'',``Buddhist'', ``Catholic'',
``Christian (Non-Denominational)'', ``Christian (Other
Denominations)'',``Hindu'', ``Jewish'', ``Muslim'',``Presbyterian,
Congregational, Reformed'', ``Other Religions''.
(\citeproc{ref-sibley2021}{Sibley, 2021})

\paragraph{Religious Identification}\label{religious-identification}

\emph{How important is your religion to how you see yourself?}

Ordinal response: (1 = Not Important, 7 = Very Important)
(\citeproc{ref-sibley2021}{Sibley, 2021})

\paragraph{Rural Classification}\label{rural-classification}

\emph{High Urban Accessibility = 1, Medium Urban Accessibility = 2, Low
Urban Accessibility = 3, Remote = 4, Very Remote = 5.}

``Participants residence locations were coded according to a five-level
ordinal categorisation ranging from Urban to Rural.''
(\citeproc{ref-whitehead2023unmasking}{Whitehead et al., 2023})

\paragraph{Sample Frame Opt in}\label{sample-frame-opt-in}

\emph{Participant was not randomly sampled from the New Zealand
Electoral Roll.}

Code string (Binary): (0 = No, 1 = Yes)
(\citeproc{ref-sibley2021}{Sibley, 2021})

\paragraph{Short Form Health}\label{short-form-health}

\emph{In general, would you say your health is\ldots{}}

Ordinal response: (1 = Poor, 7 = Excellent)
(\citeproc{ref-instrument1992mos}{Instrument Ware Jr \& Sherbourne,
1992})

\paragraph{Smoker}\label{smoker}

\emph{Do you currently smoke tobacco cigarettes?}

Binary smoking indicator (0 = No, 1 = Yes).
(\citeproc{ref-sibley2021}{Sibley, 2021})

\paragraph{Exposure Measures}\label{exposure-measures}

\subsubsection{Exposure Variable}\label{exposure-variable}

\paragraph{Extraversion}\label{extraversion}

\emph{I am the life of the party.} \emph{I don't talk a lot (reversed).}
\emph{I keep in the background (reversed).} \emph{I talk to a lot of
different people at parties.}

Mini-IPIP6 Extraversion dimension: (i) I am the life of the party. (ii)
I don't talk a lot. (r) (iii) I keep in the background. (r) (iv) I talk
to a lot of different people at parties.
(\citeproc{ref-sibley2011}{Sibley et al., 2011})

\paragraph{Outcome Measures}\label{outcome-measures}

\subsubsection{Outcome Variables}\label{outcome-variables}

\paragraph{Social Belonging}\label{social-belonging-1}

\emph{Know that people in my life accept and value me.} \emph{Feel like
an outsider (reversed).} \emph{Know that people around me share my
attitudes and beliefs.}

We assessed felt belongingness with three items adapted from the Sense
of Belonging Instrument (Hagerty \& Patusky, 1995): (1) ``Know that
people in my life accept and value me''; (2) ``Feel like an outsider'';
(3) ``Know that people around me share my attitudes and beliefs''.
Participants responded on a scale from 1 (Very Inaccurate) to 7 (Very
Accurate). The second item was reversely coded.
(\citeproc{ref-hagerty1995}{Hagerty \& Patusky, 1995})

\paragraph{Anxiety}\label{anxiety}

\emph{During the past 30 days, how often did\ldots you feel restless or
fidgety?} \emph{During the past 30 days, how often did\ldots you feel
that everything was an effort?} \emph{During the past 30 days, how often
did\ldots you feel nervous?}

Ordinal response: (0 = None Of The Time; 1 = A Little Of The Time; 2=
Some Of The Time; 3 = Most Of The Time; 4 = All Of The Time)
(\citeproc{ref-kessler2002}{Kessler et al., 2002})

\paragraph{Depression}\label{depression}

\emph{During the past 30 days, how often did\ldots you feel hopeless?}
\emph{During the past 30 days, how often did\ldots you feel so depressed
that nothing could cheer you up?} \emph{During the past 30 days, how
often did\ldots you feel you feel restless or fidgety?}

Ordinal response: (0 = None Of The Time; 1 = A Little Of The Time; 2=
Some Of The Time; 3 = Most Of The Time; 4 = All Of The Time)
(\citeproc{ref-kessler2002}{Kessler et al., 2002})

\paragraph{Life Satisfaction}\label{life-satisfaction}

\emph{I am satisfied with my life.} \emph{In most ways my life is close
to ideal.}

Ordinal response (1 = Strongly Disagree to 7 = Strongly Agree).
(\citeproc{ref-diener1985a}{Diener et al., 1985})

\paragraph{Log Hours of Exercise}\label{log-hours-of-exercise-1}

\emph{Hours spent\ldots exercising/physical activity.}

We took the natural log of the response + 1.
(\citeproc{ref-sibley2011}{Sibley et al., 2011})

\paragraph{Meaning Purpose}\label{meaning-purpose}

\emph{My life has a clear sense of purpose}

Ordinal response (1 = Strongly Disagree to 7 = Strongly Agree).
(\citeproc{ref-steger_meaning_2006}{Steger et al., 2006})

\paragraph{Meaning Sense}\label{meaning-sense}

\emph{I have a good sense of what makes my life meaningful.}

Ordinal response (1 = Strongly Disagree to 7 = Strongly Agree).
(\citeproc{ref-steger_meaning_2006}{Steger et al., 2006})

\paragraph{Neighbourhood Community}\label{neighbourhood-community}

\emph{I feel a sense of community with others in my local
neighbourhood.}

Ordinal response (1 = Strongly Disagree to 7 = Strongly Agree).
(\citeproc{ref-sengupta2013}{Sengupta et al., 2013})

\paragraph{Personal Well Being Index}\label{personal-well-being-index}

no information available for this variable.

\paragraph{Rumination}\label{rumination}

\emph{During the last 30 days, how often did\ldots you have negative
thoughts that repeated over and over?}

Ordinal responses: 0 = None of The Time, 1 = A little of The Time, 2 =
Some of The Time, 3 = Most of The Time, 4 = All of The Time.
(\citeproc{ref-nolen-hoeksema_effects_1993}{Nolen-hoeksema \& Morrow,
1993})

\paragraph{Self Esteem}\label{self-esteem}

\emph{On the whole am satisfied with myself.} \emph{Take a positive
attitude toward myself.} \emph{Am inclined to feel that I am a failure
(reversed).}

Ordinal response (1 = Very inaccurate to 7 = Very accurate).
(\citeproc{ref-Rosenberg1965}{Rosenberg, 1965})

\paragraph{Social Support}\label{social-support}

\emph{There are people I can depend on to help me if I really need it.}
\emph{There is no one I can turn to for guidance in times of stress
(reversed).} \emph{I know there are people I can turn to when I need
help.}

Ordinal response: (1 = Strongly Disagree, 7 = Strongly Agree)
(\citeproc{ref-cutrona1987}{Cutrona \& Russell, 1987})

\newpage{}

\subsection{Appendix B: Sample Characteristics}\label{appendix-sample}

\paragraph{Sample Statistics: Baseline
Covariates}\label{sample-statistics-baseline-covariates}

Table~\ref{tbl-appendix-baseline} presents sample demographic
statistics.

\begin{longtable}[]{@{}ll@{}}
\caption{Demographic statistics for New Zealand Attitudes and Values
Cohort:
\{baseline\_wave\_glued\}.}\label{tbl-appendix-baseline}\tabularnewline
\toprule\noalign{}
& 2018 \\
\midrule\noalign{}
\endfirsthead
\toprule\noalign{}
& 2018 \\
\midrule\noalign{}
\endhead
\bottomrule\noalign{}
\endlastfoot
& (N=39635) \\
\textbf{Age} & \\
Mean (SD) & 48.5 (13.9) \\
Median {[}Min, Max{]} & 51.0 {[}18.0, 99.0{]} \\
\textbf{Agreeableness} & \\
Mean (SD) & 5.35 (0.988) \\
Median {[}Min, Max{]} & 5.47 {[}1.00, 7.00{]} \\
Missing & 9 (0.0\%) \\
\textbf{Alcohol Frequency} & \\
Mean (SD) & 2.16 (1.34) \\
Median {[}Min, Max{]} & 2.00 {[}0, 5.00{]} \\
Missing & 1342 (3.4\%) \\
\textbf{Alcohol Intensity} & \\
Mean (SD) & 2.15 (2.09) \\
Median {[}Min, Max{]} & 2.00 {[}0, 15.0{]} \\
Missing & 2348 (5.9\%) \\
\textbf{Belong} & \\
Mean (SD) & 5.14 (1.07) \\
Median {[}Min, Max{]} & 5.31 {[}1.00, 7.00{]} \\
Missing & 7 (0.0\%) \\
\textbf{Born in NZ} & \\
0 & 8510 (21.5\%) \\
1 & 30670 (77.4\%) \\
Missing & 455 (1.1\%) \\
\textbf{Conscientiousness} & \\
Mean (SD) & 5.10 (1.06) \\
Median {[}Min, Max{]} & 5.23 {[}1.00, 7.00{]} \\
\textbf{Education Level} & \\
no\_qualification & 1003 (2.5\%) \\
cert\_1\_to\_4 & 13801 (34.8\%) \\
cert\_5\_to\_6 & 4953 (12.5\%) \\
university & 10400 (26.2\%) \\
post\_grad & 4220 (10.6\%) \\
masters & 3297 (8.3\%) \\
doctorate & 930 (2.3\%) \\
Missing & 1031 (2.6\%) \\
\textbf{Employed} & \\
0 & 8111 (20.5\%) \\
1 & 31475 (79.4\%) \\
Missing & 49 (0.1\%) \\
\textbf{Ethnicity} & \\
euro & 31454 (79.4\%) \\
maori & 4561 (11.5\%) \\
pacific & 971 (2.4\%) \\
asian & 2124 (5.4\%) \\
Missing & 525 (1.3\%) \\
\textbf{Disability Status} & \\
Mean (SD) & 0.223 (0.416) \\
Median {[}Min, Max{]} & 0 {[}0, 1.00{]} \\
Missing & 745 (1.9\%) \\
\textbf{Log Hours with Children} & \\
Mean (SD) & 1.18 (1.61) \\
Median {[}Min, Max{]} & 0.0341 {[}0, 5.13{]} \\
Missing & 1242 (3.1\%) \\
\textbf{Log Hours Commuting} & \\
Mean (SD) & 1.50 (0.832) \\
Median {[}Min, Max{]} & 1.61 {[}0, 4.40{]} \\
Missing & 1242 (3.1\%) \\
\textbf{Log Hours Exercising} & \\
Mean (SD) & 1.55 (0.846) \\
Median {[}Min, Max{]} & 1.61 {[}0, 4.40{]} \\
Missing & 1242 (3.1\%) \\
\textbf{Log Hours on Housework} & \\
Mean (SD) & 2.14 (0.782) \\
Median {[}Min, Max{]} & 2.20 {[}0, 5.13{]} \\
Missing & 1242 (3.1\%) \\
\textbf{Log Household Income} & \\
Mean (SD) & 11.4 (0.765) \\
Median {[}Min, Max{]} & 11.5 {[}0.685, 14.9{]} \\
Missing & 3067 (7.7\%) \\
\textbf{Male} & \\
0 & 24766 (62.5\%) \\
1 & 14767 (37.3\%) \\
Missing & 102 (0.3\%) \\
\textbf{Neuroticism} & \\
Mean (SD) & 3.49 (1.15) \\
Median {[}Min, Max{]} & 3.48 {[}1.00, 7.00{]} \\
Missing & 10 (0.0\%) \\
\textbf{Non-heterosexual} & \\
0 & 35100 (88.6\%) \\
1 & 2562 (6.5\%) \\
Missing & 1973 (5.0\%) \\
\textbf{NZ Deprivation Index} & \\
Mean (SD) & 4.77 (2.73) \\
Median {[}Min, Max{]} & 4.05 {[}1.00, 10.0{]} \\
Missing & 255 (0.6\%) \\
\textbf{Occupational Prestige Index} & \\
Mean (SD) & 54.1 (16.5) \\
Median {[}Min, Max{]} & 54.0 {[}10.0, 90.0{]} \\
Missing & 536 (1.4\%) \\
\textbf{Openness} & \\
Mean (SD) & 4.96 (1.12) \\
Median {[}Min, Max{]} & 5.00 {[}1.00, 7.00{]} \\
Missing & 3 (0.0\%) \\
\textbf{Parent} & \\
0 & 11539 (29.1\%) \\
1 & 27776 (70.1\%) \\
Missing & 320 (0.8\%) \\
\textbf{Has Partner} & \\
Mean (SD) & 0.752 (0.432) \\
Median {[}Min, Max{]} & 1.00 {[}0, 1.00{]} \\
Missing & 1244 (3.1\%) \\
\textbf{Political Conservatism} & \\
Mean (SD) & 3.59 (1.38) \\
Median {[}Min, Max{]} & 3.97 {[}1.00, 7.00{]} \\
Missing & 2682 (6.8\%) \\
\textbf{Major Religions} & \\
not\_rel & 24886 (62.8\%) \\
anglican & 2087 (5.3\%) \\
buddist & 332 (0.8\%) \\
catholic & 3123 (7.9\%) \\
christian\_nfd & 4534 (11.4\%) \\
christian\_others & 1738 (4.4\%) \\
hindu & 206 (0.5\%) \\
jewish & 80 (0.2\%) \\
muslim & 90 (0.2\%) \\
presby\_cong\_reform & 875 (2.2\%) \\
the\_others & 1068 (2.7\%) \\
Missing & 616 (1.6\%) \\
\textbf{Religious Identification} & \\
Mean (SD) & 2.36 (2.18) \\
Median {[}Min, Max{]} & 1.00 {[}1.00, 7.00{]} \\
Missing & 1050 (2.6\%) \\
\textbf{Rural Classification} & \\
High Urban Accessibility & 24406 (61.6\%) \\
Medium Urban Accessibility & 7431 (18.7\%) \\
Low Urban Accessibility & 4818 (12.2\%) \\
Remote & 2241 (5.7\%) \\
Very Remote & 486 (1.2\%) \\
Missing & 253 (0.6\%) \\
\textbf{Sample Frame Opt-In} & \\
0 & 38485 (97.1\%) \\
1 & 1150 (2.9\%) \\
\textbf{Short Form Health} & \\
Mean (SD) & 5.05 (1.17) \\
Median {[}Min, Max{]} & 5.04 {[}1.00, 7.00{]} \\
Missing & 6 (0.0\%) \\
\textbf{Smoker} & \\
0 & 35771 (90.3\%) \\
1 & 2880 (7.3\%) \\
Missing & 984 (2.5\%) \\
\end{longtable}

\subsubsection{Sample Statistics: Exposure
Variable}\label{appendix-exposure}

\begin{longtable}[]{@{}lll@{}}
\caption{Demographic statistics for New Zealand Attitudes and Values
Cohort waves 2018.}\label{tbl-appendix-exposures}\tabularnewline
\toprule\noalign{}
& 2018 & 2019 \\
\midrule\noalign{}
\endfirsthead
\toprule\noalign{}
& 2018 & 2019 \\
\midrule\noalign{}
\endhead
\bottomrule\noalign{}
\endlastfoot
& (N=39635) & (N=39635) \\
\textbf{Extraversion} & & \\
Mean (SD) & 3.91 (1.20) & 3.86 (1.19) \\
Median {[}Min, Max{]} & 3.96 {[}1.00, 7.00{]} & 3.79 {[}1.00, 7.00{]} \\
Missing & 0 (0\%) & 11117 (28.0\%) \\
\textbf{Extraversion (binary)} & & \\
{[}1.0,4.0{]} & 21138 (53.3\%) & 15637 (39.5\%) \\
(4.0,7.0{]} & 18497 (46.7\%) & 12881 (32.5\%) \\
Missing & 0 (0\%) & 11117 (28.0\%) \\
\end{longtable}

\newpage{}

\subsubsection{Sample Statistics: Outcome
Variables}\label{appendix-outcomes}

\begin{longtable}[]{@{}
  >{\raggedright\arraybackslash}p{(\linewidth - 6\tabcolsep) * \real{0.3571}}
  >{\raggedright\arraybackslash}p{(\linewidth - 6\tabcolsep) * \real{0.2143}}
  >{\raggedright\arraybackslash}p{(\linewidth - 6\tabcolsep) * \real{0.2143}}
  >{\raggedright\arraybackslash}p{(\linewidth - 6\tabcolsep) * \real{0.2143}}@{}}
\caption{Outcome variables measured
at}\label{tbl-appendix-outcomes}\tabularnewline
\toprule\noalign{}
\begin{minipage}[b]{\linewidth}\raggedright
\end{minipage} & \begin{minipage}[b]{\linewidth}\raggedright
2018
\end{minipage} & \begin{minipage}[b]{\linewidth}\raggedright
2020
\end{minipage} & \begin{minipage}[b]{\linewidth}\raggedright
Overall
\end{minipage} \\
\midrule\noalign{}
\endfirsthead
\toprule\noalign{}
\begin{minipage}[b]{\linewidth}\raggedright
\end{minipage} & \begin{minipage}[b]{\linewidth}\raggedright
2018
\end{minipage} & \begin{minipage}[b]{\linewidth}\raggedright
2020
\end{minipage} & \begin{minipage}[b]{\linewidth}\raggedright
Overall
\end{minipage} \\
\midrule\noalign{}
\endhead
\bottomrule\noalign{}
\endlastfoot
& (N=39635) & (N=39635) & (N=79270) \\
\textbf{Social Belonging} & & & \\
Mean (SD) & 5.14 (1.07) & 5.06 (1.09) & 5.11 (1.08) \\
Median {[}Min, Max{]} & 5.31 {[}1.00, 7.00{]} & 5.05 {[}1.00, 7.00{]} &
5.30 {[}1.00, 7.00{]} \\
Missing & 7 (0.0\%) & 13278 (33.5\%) & 13285 (16.8\%) \\
\textbf{Anxiety} & & & \\
Mean (SD) & 1.21 (0.774) & 1.17 (0.756) & 1.19 (0.767) \\
Median {[}Min, Max{]} & 1.00 {[}0, 4.00{]} & 1.00 {[}0, 4.00{]} & 1.00
{[}0, 4.00{]} \\
Missing & 51 (0.1\%) & 13275 (33.5\%) & 13326 (16.8\%) \\
\textbf{Depression} & & & \\
Mean (SD) & 0.584 (0.751) & 0.550 (0.723) & 0.571 (0.740) \\
Median {[}Min, Max{]} & 0.333 {[}0, 4.00{]} & 0.333 {[}0, 4.00{]} &
0.333 {[}0, 4.00{]} \\
Missing & 54 (0.1\%) & 13273 (33.5\%) & 13327 (16.8\%) \\
\textbf{Life Satisfaction} & & & \\
Mean (SD) & 5.30 (1.20) & 5.25 (1.23) & 5.28 (1.21) \\
Median {[}Min, Max{]} & 5.50 {[}1.00, 7.00{]} & 5.50 {[}1.00, 7.00{]} &
5.50 {[}1.00, 7.00{]} \\
Missing & 260 (0.7\%) & 13560 (34.2\%) & 13820 (17.4\%) \\
\textbf{Hours of Exercise (log)} & & & \\
Mean (SD) & 1.55 (0.846) & 1.63 (0.839) & 1.58 (0.844) \\
Median {[}Min, Max{]} & 1.61 {[}0, 4.40{]} & 1.78 {[}0, 4.40{]} & 1.61
{[}0, 4.40{]} \\
Missing & 1242 (3.1\%) & 13770 (34.7\%) & 15012 (18.9\%) \\
Meaning: Purpose & & & \\
Mean (SD) & 5.20 (1.41) & 5.15 (1.44) & 5.18 (1.42) \\
Median {[}Min, Max{]} & 5.05 {[}1.00, 7.00{]} & 5.04 {[}1.00, 7.00{]} &
5.04 {[}1.00, 7.00{]} \\
Missing & 1010 (2.5\%) & 13650 (34.4\%) & 14660 (18.5\%) \\
Meaning: Sense & & & \\
Mean (SD) & 5.71 (1.22) & 5.71 (1.19) & 5.71 (1.20) \\
Median {[}Min, Max{]} & 5.99 {[}1.00, 7.00{]} & 5.99 {[}1.00, 7.00{]} &
5.99 {[}1.00, 7.00{]} \\
Missing & 128 (0.3\%) & 13162 (33.2\%) & 13290 (16.8\%) \\
\textbf{Neighbourhood Community} & & & \\
Mean (SD) & 4.19 (1.66) & 4.38 (1.57) & 4.27 (1.63) \\
Median {[}Min, Max{]} & 4.03 {[}1.00, 7.00{]} & 4.95 {[}1.00, 7.00{]} &
4.04 {[}1.00, 7.00{]} \\
Missing & 212 (0.5\%) & 13202 (33.3\%) & 13414 (16.9\%) \\
\textbf{Personal Well-being Index} & & & \\
Mean (SD) & 7.09 (1.66) & 7.18 (1.63) & 7.12 (1.65) \\
Median {[}Min, Max{]} & 7.29 {[}0, 10.0{]} & 7.47 {[}0, 10.0{]} & 7.46
{[}0, 10.0{]} \\
Missing & 41 (0.1\%) & 13120 (33.1\%) & 13161 (16.6\%) \\
\textbf{Rumination} & & & \\
Mean (SD) & 0.853 (1.00) & 0.797 (0.959) & 0.831 (0.987) \\
Median {[}Min, Max{]} & 0.955 {[}0, 4.00{]} & 0.0495 {[}0, 4.00{]} &
0.953 {[}0, 4.00{]} \\
Missing & 135 (0.3\%) & 13335 (33.6\%) & 13470 (17.0\%) \\
\textbf{Self Esteem} & & & \\
Mean (SD) & 5.14 (1.28) & 5.13 (1.27) & 5.14 (1.28) \\
Median {[}Min, Max{]} & 5.34 {[}1.00, 7.00{]} & 5.34 {[}1.00, 7.00{]} &
5.34 {[}1.00, 7.00{]} \\
Missing & 11 (0.0\%) & 13280 (33.5\%) & 13291 (16.8\%) \\
\textbf{Social Support} & & & \\
Mean (SD) & 5.95 (1.12) & 5.94 (1.12) & 5.95 (1.12) \\
Median {[}Min, Max{]} & 6.30 {[}1.00, 7.00{]} & 6.29 {[}1.00, 7.00{]} &
6.30 {[}1.00, 7.00{]} \\
Missing & 30 (0.1\%) & 13112 (33.1\%) & 13142 (16.6\%) \\
\end{longtable}

\newpage{}

\subsection{Appendix C: Transition Matrix to Check The Positivity
Assumption}\label{appendix-transition}

\begin{longtable}[]{@{}lrrr@{}}
\caption{Transition Matrix Showing
Change}\label{tbl-transition}\tabularnewline
\toprule\noalign{}
From / To & State 0 & State 1 & Total \\
\midrule\noalign{}
\endfirsthead
\toprule\noalign{}
From / To & State 0 & State 1 & Total \\
\midrule\noalign{}
\endhead
\bottomrule\noalign{}
\endlastfoot
State 0 & 17572 & 2271 & 19843 \\
State 1 & 2400 & 6275 & 8675 \\
\end{longtable}

These transition matrices capture shifts in states between consecutive
waves. Each cell shows the count of individuals transitioning from one
state to another. Rows are the initial state (From), columns the
subsequent state (To). \textbf{Diagonal entries} (in \textbf{bold}) mark
those who stayed in the same state.

\newpage{}

\subsection{Appendix D: RATE AUTOC and RATE Qini}\label{appendix-rate}

Refer to \hyperref[appendix-cate-validation]{Appendix D} for details.

\subparagraph{RATE AUTOC RESULTS}\label{rate-autoc-results}

\subsubsection{Evidence for heterogeneous treatment effects (policy =
treat best responders) using
AUTOC}\label{evidence-for-heterogeneous-treatment-effects-policy-treat-best-responders-using-autoc}

AUTOC uses logarithmic weighting to focus treatment on top responders.

Note: The following outcomes were inverted during preprocessing because
higher values of the exposure correspond to worse outcomes: Anxiety,
Depression, Rumination.

Positive RATE estimates for: \textbf{Hours of Exercise (log)}.

Estimates (\textbf{Hours of Exercise (log)}: 0.084 (95\% CI 0.035,
0.133)) show robust heterogeneity.

Negative RATE estimates for: Neighbourhood Community.

Estimates (Neighbourhood Community: -0.076 (95\% CI -0.137, -0.015))
caution against CATE prioritisation.

For outcomes with adjusted p-values not meeting the FDR threshold of q =
0.20 (Meaning: Sense, Anxiety (reversed), Rumination (reversed), Self
Esteem, Social Support, Life Satisfaction, Meaning: Purpose, Depression
(reversed), Social Belonging, Personal Well-being Index), evidence is
inconclusive.

\textbf{?@fig-rate-1} presents the RATE AUTOC curve for \textbf{Hours of
Exercise (log)}

\newpage{}

\subsection{Appendix E. Estimating and Interpreting Heterogeneous
Treatment Effects with GRF}\label{appendix-explain-grf}

\newpage{}

\subsection{Appendix F: Strengths and Limitations of Causal
Forests}\label{appendix-rate}

\newpage{}

\subsection*{References}\label{references}
\addcontentsline{toc}{subsection}{References}

\phantomsection\label{refs}
\begin{CSLReferences}{1}{0}
\bibitem[\citeproctext]{ref-atkinson2019}
Atkinson, J., Salmond, C., \& Crampton, P. (2019). \emph{NZDep2018 index
of deprivation, user{'}s manual.}

\bibitem[\citeproctext]{ref-cutrona1987}
Cutrona, C. E., \& Russell, D. W. (1987). The provisions of social
relationships and adaptation to stress. \emph{Advances in Personal
Relationships}, \emph{1}, 37--67.

\bibitem[\citeproctext]{ref-diener1985a}
Diener, E., Emmons, R. A., Larsen, R. J., \& Griffin, S. (1985). The
Satisfaction With Life Scale. \emph{Journal of Personality Assessment},
\emph{49}(1), 71--75.

\bibitem[\citeproctext]{ref-fahy2017}
Fahy, K. M., Lee, A., \& Milne, B. J. (2017). \emph{{N}ew {Z}ealand
socio-economic index 2013}. Statistics New Zealand-Tatauranga Aotearoa.

\bibitem[\citeproctext]{ref-fraser_coding_2020}
Fraser, G., Bulbulia, J., Greaves, L. M., Wilson, M. S., \& Sibley, C.
G. (2020). Coding responses to an open-ended gender measure in a {N}ew
{Z}ealand national sample. \emph{The Journal of Sex Research},
\emph{57}(8), 979--986.
\url{https://doi.org/10.1080/00224499.2019.1687640}

\bibitem[\citeproctext]{ref-greaves2017diversity}
Greaves, L. M., Barlow, F. K., Lee, C. H., Matika, C. M., Wang, W.,
Lindsay, C.-J., Case, C. J., Sengupta, N. K., Huang, Y., Cowie, L. J.,
et al. (2017). The diversity and prevalence of sexual orientation
self-labels in a {N}ew {Z}ealand national sample. \emph{Archives of
Sexual Behavior}, \emph{46}, 1325--1336.

\bibitem[\citeproctext]{ref-hagerty1995}
Hagerty, B. M. K., \& Patusky, K. (1995). Developing a Measure Of Sense
of Belonging: \emph{Nursing Research}, \emph{44}(1), 9--13.
\url{https://doi.org/10.1097/00006199-199501000-00003}

\bibitem[\citeproctext]{ref-Ministry_of_Health_2013}
Health, Ministry of. (2013). \emph{The {N}ew {Z}ealand {H}ealth
{S}urvey: Content guide 2012-2013}. Princeton University Press.

\bibitem[\citeproctext]{ref-instrument1992mos}
Instrument Ware Jr, J., \& Sherbourne, C. (1992). The MOS 36-item
short-form health survey (SF-36): I. Conceptual framework and item
selection. \emph{Medical Care}, \emph{30}(6), 473--483.

\bibitem[\citeproctext]{ref-jost_end_2006-1}
Jost, J. T. (2006). The end of the end of ideology. \emph{American
Psychologist}, \emph{61}(7), 651--670.
\url{https://doi.org/10.1037/0003-066X.61.7.651}

\bibitem[\citeproctext]{ref-kessler2002}
Kessler, R. ~C., Andrews, G., Colpe, L. ~J., Hiripi, E., Mroczek, D.
~K., Normand, S.-L. ~T., Walters, E. ~E., \& Zaslavsky, A. ~M. (2002).
Short screening scales to monitor population prevalences and trends in
non-specific psychological distress. \emph{Psychological Medicine},
\emph{32}(6), 959--976. \url{https://doi.org/10.1017/S0033291702006074}

\bibitem[\citeproctext]{ref-nolen-hoeksema_effects_1993}
Nolen-hoeksema, S., \& Morrow, J. (1993). Effects of rumination and
distraction on naturally occurring depressed mood. \emph{Cognition and
Emotion}, \emph{7}(6), 561--570.
\url{https://doi.org/10.1080/02699939308409206}

\bibitem[\citeproctext]{ref-Rosenberg1965}
Rosenberg, M. (1965). \emph{Society and the adolescent self-image}.
Princeton University Press.

\bibitem[\citeproctext]{ref-sengupta2013}
Sengupta, N. K., Luyten, N., Greaves, L. M., Osborne, D., Robertson, A.,
Brunton, C., Armstrong, G., \& Sibley, C. G. (2013). Sense of Community
in {N}ew {Z}ealand Neighbourhoods: A Multi-Level Model Predicting Social
Capital. \emph{New Zealand Journal of Psychology}, \emph{42}(1), 36--45.

\bibitem[\citeproctext]{ref-sibley2021}
Sibley, C. G. (2021). \emph{Sampling procedure and sample details for
the {N}ew {Z}ealand {A}ttitudes and {V}alues {S}tudy}.
\url{https://doi.org/10.31234/osf.io/wgqvy}

\bibitem[\citeproctext]{ref-sibley2011}
Sibley, C. G., Luyten, N., Purnomo, M., Mobberley, A., Wootton, L. W.,
Hammond, M. D., Sengupta, N., Perry, R., West-Newman, T., Wilson, M. S.,
McLellan, L., Hoverd, W. J., \& Robertson, A. (2011). The Mini-IPIP6:
Validation and extension of a short measure of the Big-Six factors of
personality in {N}ew {Z}ealand. \emph{New Zealand Journal of
Psychology}, \emph{40}(3), 142--159.

\bibitem[\citeproctext]{ref-statsnz_ssga18}
Statistics New Zealand. (2017). \emph{Statistical standard for
geographic areas 2018 (SSGA18)}. Statistics New Zealand.
\url{https://www.stats.govt.nz/methods/statistical-standard-for-geographic-areas-2018/}

\bibitem[\citeproctext]{ref-steger_meaning_2006}
Steger, M. F., Frazier, P., Oishi, S., \& Kaler, M. (2006). The meaning
in life questionnaire: Assessing the presence of and search for meaning
in life. \emph{Journal of Counseling Psychology}, \emph{53}(1), 80--93.
\url{https://doi.org/10.1037/0022-0167.53.1.80}

\bibitem[\citeproctext]{ref-verbrugge1997}
Verbrugge, L. M. (1997). A global disability indicator. \emph{Journal of
Aging Studies}, \emph{11}(4), 337--362.
\url{https://doi.org/10.1016/S0890-4065(97)90026-8}

\bibitem[\citeproctext]{ref-whitehead2023unmasking}
Whitehead, J., Davie, G., Graaf, B. de, Crengle, S., Lawrenson, R.,
Miller, R., \& Nixon, G. (2023). Unmasking hidden disparities: A
comparative observational study examining the impact of different
rurality classifications for health research in aotearoa new zealand.
\emph{BMJ Open}, \emph{13}(4), e067927.

\end{CSLReferences}




\end{document}
